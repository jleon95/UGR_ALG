\documentclass[11pt,a4paper]{article}
\usepackage[utf8]{inputenc}
\usepackage{color}
\usepackage{enumerate}
\usepackage{fancyhdr}
\usepackage{minted}
\usepackage{graphicx}
\usepackage{array}
\usepackage{hyperref}
\usepackage{amssymb}
\usepackage[spanish]{babel}
\usepackage[spanish]{algorithm2e}

\setlength{\oddsidemargin}{18pt}
\setlength{\headheight}{14pt}
\setlength{\textheight}{609pt}
\setlength{\marginparsep}{11pt}
\setlength{\footskip}{30pt}
\hoffset = 0pt
\voffset = 0pt
\setlength{\topmargin}{0pt}
\setlength{\headsep}{25pt}
\setlength{\textwidth}{424pt}
\setlength{\marginparwidth}{54pt}
\setlength{\marginparpush}{5pt}
\paperwidth = 597pt
\paperheight = 845pt

\pagestyle{fancy}
\fancyhead[LO]{\textcolor[rgb]{0,0,0}{Grado en Ingeniería Informática}}
\fancyhead[RO]{\textcolor[rgb]{0.2,0.2,0.9}{Algorítmica, Curso 2015-2016}}

\hypersetup{
	colorlinks,
	citecolor=black,
	filecolor=black,
	linkcolor=black,
	urlcolor=black
}

\begin{document}

	\begin{titlepage}

		\centering

		\begin{figure}[h]

			\centering
			\includegraphics[width=0.6\textwidth]{logo-ugr.png}
			
		\end{figure}

		\vspace{1cm}

		{\scshape\LARGE Universidad de Granada}

		\vspace{1cm}

		{\LARGE Algorítmica}

		\vspace{1cm}

		{\huge\bfseries\textit{Algoritmos Greedy, parte 1}}

		\vspace{1cm}

		{\itshape\large 
		Laura Calle Caraballo \\
		Cristina María Garrido López \\
		Germán González Almagro \\
		Javier León Palomares \\
		Antonio Manuel Milán Jiménez}

		\vfill

		{\Large\today}

	\end{titlepage}

\newpage

	\tableofcontents

\newpage

	\section{Introducción.}

		\subsection{Descripción del problema.}

			\par
			Un granjero necesita disponer siempre de un determinado fertilizante. La cantidad máxima que puede almacenar la consume en $r$ días y, antes de que eso ocurra, necesita acudir a una tienda del pueblo para abastecerse. El problema es que dicha tienda tiene un horario de apertura muy irregular (sólo abre determinados días). El granjero conoce los días en que abre la tienda, y desea minimizar el número de desplazamientos al pueblo para abastecerse.

		\subsection{Resolución.}

			\par
			Se ha diseñado un algoritmo \textit{Greedy} para determinar los días en los que debe ir al pueblo a abastecerse durante un cierto intervalo de tiempo. Asimismo, se demuestra que las soluciones proporcionadas por este algoritmo son óptimas.

	\section{Algoritmo desarrollado.}

		\subsection{Elementos.}

			\begin{itemize}

				\item
				Conjunto de Candidatos: lista de días que abre la tienda.
				\item
				Conjunto de Seleccionados: lista de días que ha ido a la tienda.
				\item
				Función Solución: ha cubierto el período de tiempo para el que tenía información.
				\item
				Función de Factibilidad: si entre dos días de apertura pasan más de $r$ días, el problema no tiene solución.
				\item
				Función Selección: puede o no esperar al próximo día que abra la tienda.
				\item
				Función Objetivo: cuántos días ha abierto la tienda.

			\end{itemize}

		\subsection{Pseudocódigo.}

				\begin{algorithm}[H]

					\textbf{function} Greedy(listaApertura, limiteDias);

					\Begin{

						ultimoDiaCompra $\longleftarrow 0$\;
						listaDias $\longleftarrow \varnothing$\;

						\For{$i = 1$ \textbf{to} $n-1$}{

							\If{listaApertura[$i+1$] - ultimoDiaCompra $>$ limiteDias}{

								listaDias $\longleftarrow$ listaDias $\cup$ listaApertura[$i$]\;
								ultimoDiaCompra $\longleftarrow$ listaApertura[$i$]\;

							}
						}

						\KwRet listaDias, $\left|listaDias\right|$\;

					}

				\end{algorithm}

\newpage

	\section{Demostración de optimalidad.}

		\par
		Suponemos una solución óptima \textit{B}, lo que quiere decir que resolverá el problema en menos visitas al proveedor que nuestra solución \textit{A}. Llamaremos $k$ al máximo valor posible donde ambas soluciones son iguales, es decir, $b(k) = a(k)$.

		\par
		Asumiendo que:

		\begin{enumerate}

			\item
			$0 = a(0) < a(1) < ... < a(p) = $ \textit{SN-Óptima}, siendo $p$ el último día seleccionado por el algoritmo. Asumimos que esta solución no es óptima.

			\item
			Sean las soluciones óptimas $0 = b(0) < b(1) < ... < b(q)$, $q<p$. Llamaremos $k$ al máximo valor posible donde $b(0) = a(0)$, $b(1) = a(1)$, ..., $b(k) = a(k)$ y sea \textit{S-Óptima} una de estas soluciones óptimas. 

		\end{enumerate}

		\par
		Entonces, sabemos que:

		\begin{enumerate}

			\item
			$a(k+1) > b(k+1)$, ya que el algoritmo \textit{Greedy} siempre escogerá el día más alejado.
			\item
			$0 = a(0) < ... < a(k) < a(k+1) < b(k+2) < ... < b(q)$, donde $q$ es el último día, sería otra solución al problema; igualmente, sería óptima, puesto que tiene el mismo tamaño que la solución óptima \textit{S-Óptima} y en la que, además, la solución \textit{A} y la nueva solución coinciden hasta $k+1$ y no hasta $k$.

			\item
			Llegamos a una contradicción, puesto que $k$ no es el máximo valor en el que $a(k)$ y $b(k)$ son iguales.

		\end{enumerate}

	\section{Conclusión.}

		\par
		Hemos podido comprobar que para este tipo de problemas un algoritmo \textit{Greedy} resulta la mejor opción ya que resuelve el problema de manera óptima. Además presenta un orden de
		eficiencia O($n$), por lo que podemos decir que es bastante eficiente.

\end{document}
